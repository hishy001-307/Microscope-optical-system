\documentclass[twocolumn]{jsarticle} %articleではなくjsarticleにすることで、「参考文献」と日本語で出る.
%\usepackage{graphicx} % Required for inserting images
\usepackage{amsfonts}
\usepackage{amsmath}
\usepackage{amssymb, amsfonts, latexsym, mathtools}
%これ3つセットで正しくpngが表示される!
\usepackage[dvipdfmx]{graphicx}
\usepackage[dvipdfmx]{color}
\usepackage[dvipdfmx]{geometry}

\usepackage{enumitem}
\usepackage{titlesec}
\usepackage{caption}
\usepackage{setspace}
\usepackage{here}
\usepackage{physics}
\usepackage{bm}
\usepackage{url} %urlをじか張りできる。
\usepackage{booktabs}
\usepackage{array}
\usepackage{color}
\usepackage{subcaption}
%\usepackage{tikz}
%\usepackage{tikz-timing}[2009/05/15]

\renewcommand{\labelenumi}{(\arabic{enumi})}


\title{物理学実験II メスバウアー効果}
\author{05251559 菱伊勇介}
\date{実験日: 2025年1月15, 19 \:共同実験者: 山口祐熙 \\レポート提出日: \today}

\begin{document}
\maketitle

\section{目的}


本実験の目的は、$\gamma$線の無反跳共鳴吸収であるメスバウアー効果を測定し、その原理を理解することである。ドップラー効果を利用して、原子核準位の自然幅および固体中の内部磁場を評価する。

本実験では、$^{57}\mathrm{Fe}$を含む吸収体の吸収スペクトルを測定する。ステンレス鋼箔から第一励起準位の自然幅を求め、鉄箔のゼーマン分裂から内部磁場および原子核の磁気モーメントを決定する。

また、本実験は測定に伴う系統誤差の評価とその影響の検討のを練習するという、教育的な一面も持つ。

\section{実験原理}
原子核が$\gamma$線を放出・吸収する際、自由原子核では反跳エネルギーのために共鳴吸収は起こりにくい。
一方、固体中では原子核が結晶格子に束縛されており、反跳運動量を格子全体で受け持つことが可能となる。
このとき反跳エネルギーは事実上無視でき、放出された$\gamma$線が原子核によって共鳴吸収される。
この現象がメスバウアー効果であり、極めて狭いエネルギー幅(今回用いた$^{57}\mathrm{Fe}$では$\frac{\Gamma}{E}\approx 10^{-12}$)をもつ共鳴吸収を観測できる点に特徴がある。

本実験では、線源と吸収体の相対速度を変化させることで、ドップラー効果を利用して$\gamma$線のエネルギーを連続的に掃引する。
相対速度$v$に対するエネルギー変化は一次の近似で$\Delta E / E = v / c$と表されるため、速度を精密に制御することで微小なエネルギー差を測定できる。

本実験で用いた single channel 方式では、吸収体を一定速度で運動させ、その速度における透過$\gamma$線の計数率を測定する。%計数率とは「カウント」のこと。

\section{実験方法}

\section{結果}

\section{考察}

\section{結論}

\clearpage
\section{メスバウアー効果の応用例}

\end{document}

platex main.tex
dvipdfmx main.dvi